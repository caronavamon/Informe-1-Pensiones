% Options for packages loaded elsewhere
\PassOptionsToPackage{unicode}{hyperref}
\PassOptionsToPackage{hyphens}{url}
%
\documentclass[
]{article}
\usepackage{amsmath,amssymb}
\usepackage{iftex}
\ifPDFTeX
  \usepackage[T1]{fontenc}
  \usepackage[utf8]{inputenc}
  \usepackage{textcomp} % provide euro and other symbols
\else % if luatex or xetex
  \usepackage{unicode-math} % this also loads fontspec
  \defaultfontfeatures{Scale=MatchLowercase}
  \defaultfontfeatures[\rmfamily]{Ligatures=TeX,Scale=1}
\fi
\usepackage{lmodern}
\ifPDFTeX\else
  % xetex/luatex font selection
\fi
% Use upquote if available, for straight quotes in verbatim environments
\IfFileExists{upquote.sty}{\usepackage{upquote}}{}
\IfFileExists{microtype.sty}{% use microtype if available
  \usepackage[]{microtype}
  \UseMicrotypeSet[protrusion]{basicmath} % disable protrusion for tt fonts
}{}
\makeatletter
\@ifundefined{KOMAClassName}{% if non-KOMA class
  \IfFileExists{parskip.sty}{%
    \usepackage{parskip}
  }{% else
    \setlength{\parindent}{0pt}
    \setlength{\parskip}{6pt plus 2pt minus 1pt}}
}{% if KOMA class
  \KOMAoptions{parskip=half}}
\makeatother
\usepackage{xcolor}
\usepackage[margin=1in]{geometry}
\usepackage{graphicx}
\makeatletter
\def\maxwidth{\ifdim\Gin@nat@width>\linewidth\linewidth\else\Gin@nat@width\fi}
\def\maxheight{\ifdim\Gin@nat@height>\textheight\textheight\else\Gin@nat@height\fi}
\makeatother
% Scale images if necessary, so that they will not overflow the page
% margins by default, and it is still possible to overwrite the defaults
% using explicit options in \includegraphics[width, height, ...]{}
\setkeys{Gin}{width=\maxwidth,height=\maxheight,keepaspectratio}
% Set default figure placement to htbp
\makeatletter
\def\fps@figure{htbp}
\makeatother
\setlength{\emergencystretch}{3em} % prevent overfull lines
\providecommand{\tightlist}{%
  \setlength{\itemsep}{0pt}\setlength{\parskip}{0pt}}
\setcounter{secnumdepth}{-\maxdimen} % remove section numbering
\usepackage[spanish, provide=*]{babel}
\usepackage{titling}
\usepackage{fontspec}
\usepackage{wrapfig}
\setmainfont{Arial}
\usepackage{xcolor}
\definecolor{azul}{RGB}{49, 126, 153 }
\definecolor{gris}{RGB}{12, 76, 99}
\usepackage{sectsty}
\sectionfont{\color{azul}}
\subsectionfont{\color{gris}}
\usepackage{longtable}
\usepackage[table]{xcolor}
\usepackage{geometry}
\usepackage{array}
\usepackage{makecell}
\usepackage{multirow}
\geometry{a4paper, margin=1in}
\definecolor{turquoise}{RGB}{0, 139, 139}
\usepackage{graphicx}
\usepackage{caption}
\captionsetup{singlelinecheck=off}
\ifLuaTeX
  \usepackage{selnolig}  % disable illegal ligatures
\fi
\usepackage{bookmark}
\IfFileExists{xurl.sty}{\usepackage{xurl}}{} % add URL line breaks if available
\urlstyle{same}
\hypersetup{
  pdftitle={Informe Metodología},
  pdfauthor={Maria Carolina Navarro Monge C05513; Tábata Picado Carmona C05961; Jimena Marchena Mendoza B74425; Valentin Chavarría Ubeda B82098},
  hidelinks,
  pdfcreator={LaTeX via pandoc}}

\title{Informe Metodología}
\usepackage{etoolbox}
\makeatletter
\providecommand{\subtitle}[1]{% add subtitle to \maketitle
  \apptocmd{\@title}{\par {\large #1 \par}}{}{}
}
\makeatother
\subtitle{Grupo 3}
\author{Maria Carolina Navarro Monge C05513 \and Tábata Picado Carmona
C05961 \and Jimena Marchena Mendoza B74425 \and Valentin Chavarría Ubeda
B82098}
\date{}

\begin{document}
\maketitle

{
\setcounter{tocdepth}{2}
\tableofcontents
}
\newpage

\section{Supuestos}\label{supuestos}

En esta sección se detallan los supuestos utilizados en la evaluación
actuarial.

\subsection{Parámetros financieros}\label{paruxe1metros-financieros}

\subsubsection{Tasa de inflación}\label{tasa-de-inflaciuxf3n}

Se asume una inflación del 4\% ya que se tiene un promedio de inflación
interanual del 4.44\%, además, como se menciona en el Informe de
política monetaria (2024), el BCCR tiene establecido como límite
superior del rango de tolerancia de inflación el 4\%.

\subsubsection{Factor de revalorización de
pensión}\label{factor-de-revalorizaciuxf3n-de-pensiuxf3n}

Los montos de pensión se actualizan 1 vez al año, específicamente en el
mes de enero, con un factor de revalorización de pensiones igual al
porcentaje de inflación del periodo (4\%)

\subsubsection{Tasa de rendimiento}\label{tasa-de-rendimiento}

\subsubsection{Gastos administrativos}\label{gastos-administrativos}

Por la información brindada se asume que el fondo no posee gastos
administrativos.

\subsection{Densidad de cotización}\label{densidad-de-cotizaciuxf3n}

Se va a asumir que las personas pertenecientes al régimen siguen el
cuadro \ref{tab:cotizaciones} de cotizaciones según la edad.

\begin{longtable}{|c|c|c|c|}
    \caption{Cantidad de cotizaciones por edad} \label{tab:cotizaciones} \\
    \hline
  \rowcolor{turquoise}
  \textbf{Edad} & \textbf{Cotizaciones} & \textbf{Edad} & \textbf{Cotizaciones} \\
  \hline
  \endfirsthead

  \hline
  \rowcolor{turquoise}
  \textbf{Edad (x)} & \textbf{Variación $(j_x)$} & \textbf{Edad (x)} & \textbf{Variación $(j_x)$} \\
  \hline
  \endhead

  \hline \multicolumn{4}{|r|}{{Continúa en la próxima página}} \\ \hline
  \endfoot

  \hline
  \endlastfoot

  % Aquí se incluyen las filas de datos
  19 & 3 & 49 & 10 \\

  20 & 5 & 50 & 10 \\

  21 & 6 & 51 & 10 \\

  22 & 7 & 52 & 10 \\

  23 & 7 & 53 & 10 \\

  24 & 8 & 54 & 10 \\

  25 & 8 & 55 & 10 \\

  26 & 9 & 56 & 10 \\

  27 & 9 & 57 & 10 \\

  28 & 9 & 58 & 10 \\

  29 & 10 & 59 & 10 \\

  30 & 10 & 60 & 10 \\

  31 & 10 & 61 & 10 \\

  32 & 10 & 62 & 11 \\

  33 & 10 & 63 & 10 \\

  34 & 10 & 64 & 10 \\

  35 & 10 & 65 & 11 \\

  36 & 10 & 66 & 10 \\

  37 & 10 & 67 & 10 \\

  38 & 10 & 68 & 11 \\

  39 & 10 & 69 & 10 \\

  40 & 10 & 70 & 11 \\

  41 & 10 & 71 & 10 \\

  42 & 10 & 72 & 11 \\

  43 & 10 & 73 & 10 \\

  44 & 10 & 74 & 11 \\

  45 & 10 & 75 & 10 \\

  46 & 10 & 76 & 10 \\

  47 & 10 & 77 & 11 \\

  48 & 11 & 78 & 10 \\
  \hline
  
    \caption*{Fuente: Elaboración propia a partir de datos del Fondo C.} 
\end{longtable}

\subsection{Postergación}\label{postergaciuxf3n}

Por la información brindada se asume que la probabilidad de que un
trabajador postergue sin importar la edad es del 10\%, es decir que
trabaje un año más después de alcanzado la edad y cuotas para
pensionarse por vejez, no aplica para otros retiros.

\subsection{Escala Salarial}\label{escala-salarial}

Se asume que las personas aumentan su salario una vez al año por medio
de la escala salarial, la cual está compuesta por el factor
inflacionario debido al ajuste al costo de vida y un factor de
antiguedad definido por la edad del asegurado. Entonces el salario de
una persona a la edad x va a estar dado por:

\[S_x = S_{x-1}(1+i)(1+j_x)\]

donde \(S_x\) representa el salario a la edad \(x\), \(i\) es el
porcentaje de inflación y \(j_x\) el porcentaje de variación del salario
por antiguedad para la edad \(x\). Este factor de antiguedad viene
definido por edad en el cuadro \ref{tab:salario}.

\begin{longtable}{|c|c|c|c|}
    \caption{Factor de antiguedad por edad} \label{tab:salario} \\
    \hline
  \rowcolor{turquoise}
  \textbf{Edad (x)} & \textbf{Variación $(j_x)$} & \textbf{Edad (x)} & \textbf{Variación $(j_x)$} \\
  \hline
  \endfirsthead

  \hline
  \rowcolor{turquoise}
  \textbf{Edad (x)} & \textbf{Variación $(j_x)$} & \textbf{Edad (x)} & \textbf{Variación $(j_x)$} \\
  \hline
  \endhead

  \hline \multicolumn{4}{|r|}{{Continúa en la próxima página}} \\ \hline
  \endfoot

  \hline
  \endlastfoot

  % Aquí se incluyen las filas de datos
  19 & 0 & 49 & 0.0119 \\

  20 & 0.0107 & 50 & -0.0069 \\

  21 & 0.2361 & 51 & -0.0027 \\

  22 & 0.1706 & 52 & -0.0111 \\

  23 & 0.1916 & 53 & -0.0186 \\

  24 & 0.1867 & 54 & 0.0066 \\

  25 & 0.1871 & 55 & -0.0042 \\

  26 & 0.1580 & 56 & -0.0139 \\

  27 & 0.1363 & 57 & -0.0020 \\

  28 & 0.1246 & 58 & -0.0168 \\

  29 & 0.1001 & 59 & -0.0097 \\

  30 & 0.0828 & 60 & -0.0210 \\

  31 & 0.0810 & 61 & 0.0316 \\

  32 & 0.0652 & 62 & -0.0289 \\

  33 & 0.0616 & 63 & -0.0702 \\

  34 & 0.0488 & 64 & 0.0260 \\

  35 & 0.0488 & 65 & -0.0587 \\

  36 & 0.0427 & 66 & 0.0664 \\

  37 & 0.0293 & 67 & -0.0700 \\

  38 & 0.0349 & 68 & 0.1230 \\

  39 & 0.0187 & 69 & 0.01363 \\

  40 & 0.0234 & 70 & 0.2840 \\

  41 & 0.0209 & 71 & 0.2174 \\

  42 & 0.0199 & 72 & -0.0889 \\

  43 & -0.0025 & 73 & 0.2306 \\

  44 & -0.0021 & 74 & 0.8912 \\

  45 & 0.0135 & 75 & 0.1452 \\

  46 & 0.0033 & 76 & -0.2793 \\

  47 & -0.0020 & 77 & 0.1238 \\

  48 & 0.0070 & 78 & 0.0677 \\
  \hline
  
    \caption*{Fuente: Elaboración propia a partir de datos del Fondo C.} 
\end{longtable}

\subsection{Porcentaje de cotización}\label{porcentaje-de-cotizaciuxf3n}

Por la información brindada se asume que la cotización aportada por los
trabajadores, patrono y estado como tal suma 15\% del salario de base.
En el cuadro () se muestra la distribución de la contribución.

\begin{table}[h]
    \centering
    \caption{Distribución de la contribución}
    \label{tab:contribuciones}
    \vspace{0.25cm}
    \begin{tabular}{|c|c|c|c|}
        \hline
        \rowcolor{turquoise}
        \multicolumn{4}{|c|}{\textbf{Aporte}} \\
        \hline
        \rowcolor{turquoise}
        \textbf{Trabajador} & \textbf{Patrono} & \textbf{Estado} & \textbf{Total} \\
        \hline
        6\% & 7\% & 2\% & 15\% \\
        \hline
    \end{tabular}\\
    \noindent\footnotesize{Fuente: Elaboración propia a partir de datos del Fondo C.}
\end{table}

\newpage

El propósito de esta valuación es analizar la vialidad del Fondo C. Para
esto existen un varias métricas aprobadas por la Superintendencia de
Pensiones (SUPEN). Dado que el fondo esta basado mediante una Prima
media nivelada, la cual consiste en una porción uniforme del salario de
los trabajados independientemente del extracto social en que se
encuentren. Es decir, las cotizaciones recaudas por el fondo al momento
\(Z\) están dada por: \[
C_z =  \pi_l\cdot S_z
\] Donde \(\pi_l\) es la prima legal y \(S_z\) los salarios al momento
\(Z\).

Con esto lo principal a analizar es la solvencia del Fondo C. Para ello
se pretende utilizar la Prima Media Teórica, el Valor Presente Actuarial
de los Beneficios Futuros (VPABF); incluyendo los pensionados en curso
de pago y pensionados futuros provenientes de la actual generación de
trabajadores actuales, Valor Presente Actuarial de los Salarios de la
generación actual (VPABS), los cuales se relacionan mediante la
siguiente formula:

\[
\pi_t = \frac{VPABF-R}{VPAS}
\] Para calcular los valores presentes actuariales de estos montos al
momento del análisis, nos basaremos en la formula presentada por el
Bowers y otros (1997), la cual se presenta a continuación:

\[
VPA = \displaystyle\sum_{k=0}^{n-1}M_kv^{k+1/2}
\]

\end{document}
