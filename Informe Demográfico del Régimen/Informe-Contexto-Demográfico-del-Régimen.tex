% Options for packages loaded elsewhere
\PassOptionsToPackage{unicode}{hyperref}
\PassOptionsToPackage{hyphens}{url}
%
\documentclass[
]{article}
\usepackage{amsmath,amssymb}
\usepackage{iftex}
\ifPDFTeX
  \usepackage[T1]{fontenc}
  \usepackage[utf8]{inputenc}
  \usepackage{textcomp} % provide euro and other symbols
\else % if luatex or xetex
  \usepackage{unicode-math} % this also loads fontspec
  \defaultfontfeatures{Scale=MatchLowercase}
  \defaultfontfeatures[\rmfamily]{Ligatures=TeX,Scale=1}
\fi
\usepackage{lmodern}
\ifPDFTeX\else
  % xetex/luatex font selection
\fi
% Use upquote if available, for straight quotes in verbatim environments
\IfFileExists{upquote.sty}{\usepackage{upquote}}{}
\IfFileExists{microtype.sty}{% use microtype if available
  \usepackage[]{microtype}
  \UseMicrotypeSet[protrusion]{basicmath} % disable protrusion for tt fonts
}{}
\makeatletter
\@ifundefined{KOMAClassName}{% if non-KOMA class
  \IfFileExists{parskip.sty}{%
    \usepackage{parskip}
  }{% else
    \setlength{\parindent}{0pt}
    \setlength{\parskip}{6pt plus 2pt minus 1pt}}
}{% if KOMA class
  \KOMAoptions{parskip=half}}
\makeatother
\usepackage{xcolor}
\usepackage[margin=1in]{geometry}
\usepackage{graphicx}
\makeatletter
\def\maxwidth{\ifdim\Gin@nat@width>\linewidth\linewidth\else\Gin@nat@width\fi}
\def\maxheight{\ifdim\Gin@nat@height>\textheight\textheight\else\Gin@nat@height\fi}
\makeatother
% Scale images if necessary, so that they will not overflow the page
% margins by default, and it is still possible to overwrite the defaults
% using explicit options in \includegraphics[width, height, ...]{}
\setkeys{Gin}{width=\maxwidth,height=\maxheight,keepaspectratio}
% Set default figure placement to htbp
\makeatletter
\def\fps@figure{htbp}
\makeatother
\setlength{\emergencystretch}{3em} % prevent overfull lines
\providecommand{\tightlist}{%
  \setlength{\itemsep}{0pt}\setlength{\parskip}{0pt}}
\setcounter{secnumdepth}{-\maxdimen} % remove section numbering
\usepackage[spanish, provide=*]{babel}
\usepackage{titling}
\usepackage{fontspec}
\setmainfont{Arial}
\usepackage{xcolor}
\definecolor{azul}{RGB}{49, 126, 153 }
\definecolor{gris}{RGB}{12, 76, 99}
\usepackage{sectsty}
\sectionfont{\color{azul}}
\subsectionfont{\color{gris}}
\ifLuaTeX
  \usepackage{selnolig}  % disable illegal ligatures
\fi
\IfFileExists{bookmark.sty}{\usepackage{bookmark}}{\usepackage{hyperref}}
\IfFileExists{xurl.sty}{\usepackage{xurl}}{} % add URL line breaks if available
\urlstyle{same}
\hypersetup{
  pdftitle={Informe Contexto Demográfico del Régimen},
  pdfauthor={Maria Carolina Navarro Monge C05513; Tábata Picado Carmona C05961; Jimena Marchena Mendoza B74425; Valentin Chavarría Ubeda B82098},
  hidelinks,
  pdfcreator={LaTeX via pandoc}}

\title{Informe Contexto Demográfico del Régimen}
\usepackage{etoolbox}
\makeatletter
\providecommand{\subtitle}[1]{% add subtitle to \maketitle
  \apptocmd{\@title}{\par {\large #1 \par}}{}{}
}
\makeatother
\subtitle{Grupo 3}
\author{Maria Carolina Navarro Monge C05513 \and Tábata Picado Carmona
C05961 \and Jimena Marchena Mendoza B74425 \and Valentin Chavarría Ubeda
B82098}
\date{}

\begin{document}
\maketitle

{
\setcounter{tocdepth}{2}
\tableofcontents
}
\newpage

\hypertarget{anuxe1lisis-general-de-la-estructura-poblacional-del-ruxe9gimen}{%
\section{Análisis general de la Estructura Poblacional del
Régimen}\label{anuxe1lisis-general-de-la-estructura-poblacional-del-ruxe9gimen}}

El estudio actuarial se llevará a cabo utilizando como base el Fondo C.
Su sostenibilidad y eficiencia dependen en gran medida de la composición
de la población que contribuye a él. Por ello, es fundamental analizar
la composición de dicho fondo a nivel poblacional. En esta sección, nos
enfocaremos en estudiar la estructura demográfica de la población que
forma parte este régimen. A continuación, se presentará un análisis
detallado de la estructura poblacional del fondo. Para llevar a cabo
este análisis, se utilizarán datos demográficos relevantes, como la
edad, el género y actividad.

\begin{wrapfigure}[15]{r}{0.5\textwidth}

 \centering

\begin{flushright}\includegraphics{Informe-Contexto-Demográfico-del-Régimen_files/figure-latex/unnamed-chunk-2-1} \end{flushright}
 

\end{wrapfigure}

Para la fecha de corte de este análisis, se revela que el Fondo presenta
una estructura poblacional robusta y activa, con una participación
equitativa de género que refleja la diversidad y equilibrio de la
sociedad actual. El regimen está compuesto por un total de 5764
personas. De ellas, 3758 (65\%) son personas activas, lo que significa
que realizaron una contribución al régimen en los últimos 12 meses.
Además, hay 411 pensionados (7\%) y los restantes 1595 (28\%) están
inactivos. En el caso del genero se encuentra muy equilibrado siendo que
el 51\% es femenino y el 49\% masculino y sin mucha variabilidad si se
segrega de forma similar a lo anterior. Todo esto permite observar un
régimen sostenible que puede apoyar a sus pensionados de manera
efectiva. Este equilibrio entre personas activas y pensionados es
crucial para la viabilidad a largo plazo del fondo.

Gracias al gráfico es posible observar la concentración de la población
en el rango de edad productiva entre los 20 y 50 años lo cual es también
un indicador positivo, ya que sugiere que el fondo puede contar con una
base sólida de contribuyentes durante los próximos años. Estos datos son
consistentes con los hallazgos del Informe Demográfico Nacional, lo que
sugiere que las tendencias observadas en el fondo de pensiones son un
reflejo de la dinámica poblacional más amplia del país. En conjunto,
estos elementos proporcionan una perspectiva alentadora para la
sostenibilidad y el crecimiento continuo del fondo, asegurando que las
necesidades de los pensionados actuales y futuros puedan ser atendidas
con confianza.

\hypertarget{poblaciuxf3n-activa-y-pensionada}{%
\section{Población activa y
pensionada}\label{poblaciuxf3n-activa-y-pensionada}}

\hypertarget{activos}{%
\subsection{Activos}\label{activos}}

Al 31 de diciembre de 2023 hay un total de 3.758 de cotizantes activos
del fondo de pensiones, cuya distribución por sexo es la siguiente:
1.907 mujeres (50,75\%) y 1.851 hombres (49,25\%).

\includegraphics{Informe-Contexto-Demográfico-del-Régimen_files/figure-latex/unnamed-chunk-6-1.pdf}

De acuerdo con los datos proporcionados, los cotizantes abarcan un rango
de edades entre los 20 y los 78 años, con una edad promedio de 43.07
años. El gráfico muestra la distribución por edad y sexo de los
cotizantes activos, destacándose una notable concentración entre los 40
y 50 años en ambos géneros.

\includegraphics{Informe-Contexto-Demográfico-del-Régimen_files/figure-latex/unnamed-chunk-9-1.pdf}

En cuanto a la antigüedad, la media se sitúa en 12.54 años. En el
gráfico se representa la distribución por rango de edad y sexo,
evidenciando un grupo notablemente longevo de cotizantes en ambos
géneros. La mayoría de ellos tienen entre 10 y 20 años de antigüedad,
seguidos por aquellos con 5 a 10 años, y los que superan los 20 años de
antigüedad.

\includegraphics{Informe-Contexto-Demográfico-del-Régimen_files/figure-latex/unnamed-chunk-12-1.pdf}

\includegraphics{Informe-Contexto-Demográfico-del-Régimen_files/figure-latex/unnamed-chunk-13-1.pdf}

\includegraphics{Informe-Contexto-Demográfico-del-Régimen_files/figure-latex/unnamed-chunk-17-1.pdf}

\hypertarget{comportamiento-de-altas-y-bajas-de-activos-y-pensionados}{%
\section{Comportamiento de altas y bajas de activos y
pensionados}\label{comportamiento-de-altas-y-bajas-de-activos-y-pensionados}}

\hypertarget{activos-1}{%
\subsection{Activos}\label{activos-1}}

\hypertarget{pensionados}{%
\subsection{Pensionados}\label{pensionados}}

A continuación, se analizan las altas de los pensionados del Régimen en
estudio, específicamente, la cantidad de entradas de pensionados por año
y por tipo de pensión (Invalidez, Vejez y Sucesión), durante un periodo
que abarca desde el 2001 al 2023.

Además, dado que el fondo no cuenta con la fecha de salida de los
pensionados, no se analizarán las bajas.

En el gráfico \#, se muestra el comportamiento de las entradas de
pensionados.

\includegraphics{Informe-Contexto-Demográfico-del-Régimen_files/figure-latex/unnamed-chunk-19-1.pdf}

Es posible observar que para el caso de los pensionados por invalidez,
las entradas entre el 2001 y el 2014 son similares, están rondando entre
1 y 3 personas. Posteriormente, se presenta un incremento a partir del
2015, alcanzando un pico de 10 nuevos pensionados por invalidez en el
2019. Sin embargo, en los último años, se experimentó una baja en la
cantidad de ingresos siendo en el 2023 de una sola persona.

En cuanto a los nuevos pensionados por sucesión, se presentan registros
a partir del 2004. Se puede identificar una tendencia a la alta con el
mayor ingreso de pensionados por este tipo entre el 2018 y el 2022,
experimentando un pico de 20 pensionados en el 2021. No obstante, al
igual que la pensión anterior, se presentó un decrecimiento en las
entradas ingresando solo una persona en el 2023.

Con respecto a los pensionados por vejez, no se presentó ningún registro
de entrada hasta el 2010. En el periodo del 2010 al 2015, la cantidad de
nuevos pensionados por vejez estuvo entre 1 y 7 personas. En los años
siguientes aumentó la cantidad de entradas, alcanzando un valor máximo
de 48 personas en el 2022. Al igual que los otros dos tipos de
pensiones, para el 2023 se sufrió una reducción en los ingresos
recibiendo a 30 personas.

En síntesis, se identifica que en el periodo de estudio, el mayor
crecimiento de entradas de pensionados son percibidos por las pensiones
por vejez, seguido de las pensiones por sucesión y por último, las de
invalidez. Las tres mostraron una tendencia a la alta pero bajaron en el
2023.

\end{document}
