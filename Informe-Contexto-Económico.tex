% Options for packages loaded elsewhere
\PassOptionsToPackage{unicode}{hyperref}
\PassOptionsToPackage{hyphens}{url}
%
\documentclass[
]{article}
\usepackage{amsmath,amssymb}
\usepackage{iftex}
\ifPDFTeX
  \usepackage[T1]{fontenc}
  \usepackage[utf8]{inputenc}
  \usepackage{textcomp} % provide euro and other symbols
\else % if luatex or xetex
  \usepackage{unicode-math} % this also loads fontspec
  \defaultfontfeatures{Scale=MatchLowercase}
  \defaultfontfeatures[\rmfamily]{Ligatures=TeX,Scale=1}
\fi
\usepackage{lmodern}
\ifPDFTeX\else
  % xetex/luatex font selection
\fi
% Use upquote if available, for straight quotes in verbatim environments
\IfFileExists{upquote.sty}{\usepackage{upquote}}{}
\IfFileExists{microtype.sty}{% use microtype if available
  \usepackage[]{microtype}
  \UseMicrotypeSet[protrusion]{basicmath} % disable protrusion for tt fonts
}{}
\makeatletter
\@ifundefined{KOMAClassName}{% if non-KOMA class
  \IfFileExists{parskip.sty}{%
    \usepackage{parskip}
  }{% else
    \setlength{\parindent}{0pt}
    \setlength{\parskip}{6pt plus 2pt minus 1pt}}
}{% if KOMA class
  \KOMAoptions{parskip=half}}
\makeatother
\usepackage{xcolor}
\usepackage[margin=1in]{geometry}
\usepackage{color}
\usepackage{fancyvrb}
\newcommand{\VerbBar}{|}
\newcommand{\VERB}{\Verb[commandchars=\\\{\}]}
\DefineVerbatimEnvironment{Highlighting}{Verbatim}{commandchars=\\\{\}}
% Add ',fontsize=\small' for more characters per line
\usepackage{framed}
\definecolor{shadecolor}{RGB}{248,248,248}
\newenvironment{Shaded}{\begin{snugshade}}{\end{snugshade}}
\newcommand{\AlertTok}[1]{\textcolor[rgb]{0.94,0.16,0.16}{#1}}
\newcommand{\AnnotationTok}[1]{\textcolor[rgb]{0.56,0.35,0.01}{\textbf{\textit{#1}}}}
\newcommand{\AttributeTok}[1]{\textcolor[rgb]{0.13,0.29,0.53}{#1}}
\newcommand{\BaseNTok}[1]{\textcolor[rgb]{0.00,0.00,0.81}{#1}}
\newcommand{\BuiltInTok}[1]{#1}
\newcommand{\CharTok}[1]{\textcolor[rgb]{0.31,0.60,0.02}{#1}}
\newcommand{\CommentTok}[1]{\textcolor[rgb]{0.56,0.35,0.01}{\textit{#1}}}
\newcommand{\CommentVarTok}[1]{\textcolor[rgb]{0.56,0.35,0.01}{\textbf{\textit{#1}}}}
\newcommand{\ConstantTok}[1]{\textcolor[rgb]{0.56,0.35,0.01}{#1}}
\newcommand{\ControlFlowTok}[1]{\textcolor[rgb]{0.13,0.29,0.53}{\textbf{#1}}}
\newcommand{\DataTypeTok}[1]{\textcolor[rgb]{0.13,0.29,0.53}{#1}}
\newcommand{\DecValTok}[1]{\textcolor[rgb]{0.00,0.00,0.81}{#1}}
\newcommand{\DocumentationTok}[1]{\textcolor[rgb]{0.56,0.35,0.01}{\textbf{\textit{#1}}}}
\newcommand{\ErrorTok}[1]{\textcolor[rgb]{0.64,0.00,0.00}{\textbf{#1}}}
\newcommand{\ExtensionTok}[1]{#1}
\newcommand{\FloatTok}[1]{\textcolor[rgb]{0.00,0.00,0.81}{#1}}
\newcommand{\FunctionTok}[1]{\textcolor[rgb]{0.13,0.29,0.53}{\textbf{#1}}}
\newcommand{\ImportTok}[1]{#1}
\newcommand{\InformationTok}[1]{\textcolor[rgb]{0.56,0.35,0.01}{\textbf{\textit{#1}}}}
\newcommand{\KeywordTok}[1]{\textcolor[rgb]{0.13,0.29,0.53}{\textbf{#1}}}
\newcommand{\NormalTok}[1]{#1}
\newcommand{\OperatorTok}[1]{\textcolor[rgb]{0.81,0.36,0.00}{\textbf{#1}}}
\newcommand{\OtherTok}[1]{\textcolor[rgb]{0.56,0.35,0.01}{#1}}
\newcommand{\PreprocessorTok}[1]{\textcolor[rgb]{0.56,0.35,0.01}{\textit{#1}}}
\newcommand{\RegionMarkerTok}[1]{#1}
\newcommand{\SpecialCharTok}[1]{\textcolor[rgb]{0.81,0.36,0.00}{\textbf{#1}}}
\newcommand{\SpecialStringTok}[1]{\textcolor[rgb]{0.31,0.60,0.02}{#1}}
\newcommand{\StringTok}[1]{\textcolor[rgb]{0.31,0.60,0.02}{#1}}
\newcommand{\VariableTok}[1]{\textcolor[rgb]{0.00,0.00,0.00}{#1}}
\newcommand{\VerbatimStringTok}[1]{\textcolor[rgb]{0.31,0.60,0.02}{#1}}
\newcommand{\WarningTok}[1]{\textcolor[rgb]{0.56,0.35,0.01}{\textbf{\textit{#1}}}}
\usepackage{graphicx}
\makeatletter
\def\maxwidth{\ifdim\Gin@nat@width>\linewidth\linewidth\else\Gin@nat@width\fi}
\def\maxheight{\ifdim\Gin@nat@height>\textheight\textheight\else\Gin@nat@height\fi}
\makeatother
% Scale images if necessary, so that they will not overflow the page
% margins by default, and it is still possible to overwrite the defaults
% using explicit options in \includegraphics[width, height, ...]{}
\setkeys{Gin}{width=\maxwidth,height=\maxheight,keepaspectratio}
% Set default figure placement to htbp
\makeatletter
\def\fps@figure{htbp}
\makeatother
\setlength{\emergencystretch}{3em} % prevent overfull lines
\providecommand{\tightlist}{%
  \setlength{\itemsep}{0pt}\setlength{\parskip}{0pt}}
\setcounter{secnumdepth}{-\maxdimen} % remove section numbering
\usepackage[spanish, provide=*]{babel}
\usepackage{titling}
\usepackage{fontspec}
\setmainfont{Arial}
\usepackage{xcolor}
\definecolor{azul}{RGB}{49, 126, 153 }
\definecolor{gris}{RGB}{12, 76, 99}
\usepackage{sectsty}
\sectionfont{\color{azul}}
\subsectionfont{\color{gris}}
\ifLuaTeX
  \usepackage{selnolig}  % disable illegal ligatures
\fi
\IfFileExists{bookmark.sty}{\usepackage{bookmark}}{\usepackage{hyperref}}
\IfFileExists{xurl.sty}{\usepackage{xurl}}{} % add URL line breaks if available
\urlstyle{same}
\hypersetup{
  pdftitle={Informe Contexto Económico Nacional},
  pdfauthor={Maria Carolina Navarro Monge C05513; Tábata Picado Carmona C05961; Jimena Marchena Mendoza},
  hidelinks,
  pdfcreator={LaTeX via pandoc}}

\title{Informe Contexto Económico Nacional}
\author{Maria Carolina Navarro Monge C05513 \and Tábata Picado Carmona
C05961 \and Jimena Marchena Mendoza}
\date{}

\begin{document}
\maketitle

{
\setcounter{tocdepth}{2}
\tableofcontents
}
\newpage

\hypertarget{inflaciuxf3n}{%
\section{Inflación}\label{inflaciuxf3n}}

\begin{Shaded}
\begin{Highlighting}[]
\NormalTok{Inflacion }\OtherTok{\textless{}{-}} \FunctionTok{read.csv2}\NormalTok{(}\StringTok{"Inflacion.csv"}\NormalTok{)}
\NormalTok{Inflacion}\SpecialCharTok{$}\NormalTok{Fecha }\OtherTok{\textless{}{-}} \FunctionTok{as.Date}\NormalTok{(Inflacion}\SpecialCharTok{$}\NormalTok{Fecha, }\AttributeTok{format =} \StringTok{"\%d/\%m/\%Y"}\NormalTok{)}
\end{Highlighting}
\end{Shaded}

\begin{Shaded}
\begin{Highlighting}[]
\FunctionTok{ggplot}\NormalTok{(Inflacion, }\FunctionTok{aes}\NormalTok{(}\AttributeTok{x =}\NormalTok{ Fecha)) }\SpecialCharTok{+}
  \FunctionTok{geom\_line}\NormalTok{(}\FunctionTok{aes}\NormalTok{(}\AttributeTok{y =} \StringTok{\textasciigrave{}}\AttributeTok{Variacion.mensual....}\StringTok{\textasciigrave{}}\NormalTok{, }\AttributeTok{color =} \StringTok{"Variación Mensual"}\NormalTok{),}\AttributeTok{size =} \DecValTok{1}\NormalTok{) }\SpecialCharTok{+}
  \FunctionTok{geom\_line}\NormalTok{(}\FunctionTok{aes}\NormalTok{(}\AttributeTok{y =} \StringTok{\textasciigrave{}}\AttributeTok{Variacion.interanual....}\StringTok{\textasciigrave{}}\NormalTok{, }\AttributeTok{color =} \StringTok{"Variación Interanual"}\NormalTok{),}\AttributeTok{size =} \DecValTok{1}\NormalTok{) }\SpecialCharTok{+}
  \FunctionTok{geom\_line}\NormalTok{(}\FunctionTok{aes}\NormalTok{(}\AttributeTok{y =} \StringTok{\textasciigrave{}}\AttributeTok{Variacion.acumulada.....n3}\StringTok{\textasciigrave{}}\NormalTok{, }\AttributeTok{color =} \StringTok{"Variación Acumulada"}\NormalTok{),}\AttributeTok{size =} \DecValTok{1}\NormalTok{) }\SpecialCharTok{+}
  \FunctionTok{scale\_color\_manual}\NormalTok{(}\AttributeTok{values =} \FunctionTok{c}\NormalTok{(}\StringTok{"Variación Mensual"} \OtherTok{=} \StringTok{"\#2F4F4F"}\NormalTok{, }\StringTok{"Variación Interanual"} \OtherTok{=} \StringTok{"\#BCEE68"}\NormalTok{, }\StringTok{"Variación Acumulada"} \OtherTok{=} \StringTok{"cadetblue3"}\NormalTok{)) }\SpecialCharTok{+}
  \FunctionTok{labs}\NormalTok{(}\AttributeTok{title =} \StringTok{"Inflación"}\NormalTok{,}
       \AttributeTok{subtitle =} \StringTok{"Ene 2022 a Dic 2023"}\NormalTok{,}
       \AttributeTok{x =} \StringTok{"Fecha"}\NormalTok{,}
       \AttributeTok{y =} \StringTok{"Variación"}\NormalTok{,}
       \AttributeTok{color =} \StringTok{"Tipo de Variación"}\NormalTok{,}
       \AttributeTok{caption =} \FunctionTok{expression}\NormalTok{(}\FunctionTok{bold}\NormalTok{(}\StringTok{"Fuente: "}\NormalTok{) }\SpecialCharTok{*} \StringTok{"Elaboración propia a partir de datos del BCCR."}\NormalTok{)) }\SpecialCharTok{+}
  \FunctionTok{theme\_minimal}\NormalTok{()}\SpecialCharTok{+}
  \FunctionTok{scale\_y\_continuous}\NormalTok{(}\AttributeTok{labels =}\NormalTok{ scales}\SpecialCharTok{::}\FunctionTok{percent\_format}\NormalTok{(}\AttributeTok{scale =} \DecValTok{1}\NormalTok{), }\AttributeTok{breaks =} \FunctionTok{seq}\NormalTok{(}\SpecialCharTok{{-}}\DecValTok{4}\NormalTok{, }\DecValTok{13}\NormalTok{, }\AttributeTok{by =} \DecValTok{1}\NormalTok{))}\SpecialCharTok{+}
  \FunctionTok{geom\_hline}\NormalTok{(}\AttributeTok{yintercept =} \DecValTok{2}\NormalTok{, }\AttributeTok{linetype =} \StringTok{"dashed"}\NormalTok{, }\AttributeTok{color =} \StringTok{"black"}\NormalTok{)}\SpecialCharTok{+}
  \FunctionTok{geom\_hline}\NormalTok{(}\AttributeTok{yintercept =} \DecValTok{4}\NormalTok{, }\AttributeTok{linetype =} \StringTok{"dashed"}\NormalTok{, }\AttributeTok{color =} \StringTok{"black"}\NormalTok{)}\SpecialCharTok{+}
  \FunctionTok{scale\_x\_date}\NormalTok{(}\AttributeTok{date\_labels =} \StringTok{"\%b{-}\%Y"}\NormalTok{, }\AttributeTok{date\_breaks =} \StringTok{"2 month"}\NormalTok{)}\SpecialCharTok{+}
  \FunctionTok{theme}\NormalTok{(}\AttributeTok{axis.text.x =} \FunctionTok{element\_text}\NormalTok{(}\AttributeTok{angle =} \DecValTok{45}\NormalTok{, }\AttributeTok{hjust =} \DecValTok{1}\NormalTok{),}
        \AttributeTok{legend.position =} \StringTok{"bottom"}\NormalTok{,}
        \AttributeTok{plot.caption.position =} \StringTok{"plot"}\NormalTok{,}
        \AttributeTok{plot.caption =} \FunctionTok{element\_text}\NormalTok{(}\AttributeTok{hjust =} \DecValTok{0}\NormalTok{),}
        \AttributeTok{plot.title =} \FunctionTok{element\_text}\NormalTok{(}\AttributeTok{hjust =} \FloatTok{0.5}\NormalTok{),}
        \AttributeTok{plot.subtitle =} \FunctionTok{element\_text}\NormalTok{(}\AttributeTok{hjust =} \FloatTok{0.5}\NormalTok{))}
\end{Highlighting}
\end{Shaded}

\begin{verbatim}
## Warning: Using `size` aesthetic for lines was deprecated in ggplot2 3.4.0.
## i Please use `linewidth` instead.
## This warning is displayed once every 8 hours.
## Call `lifecycle::last_lifecycle_warnings()` to see where this warning was
## generated.
\end{verbatim}

\includegraphics{Informe-Contexto-Económico_files/figure-latex/unnamed-chunk-3-1.pdf}

En enero de 2022, el país experimentó una inflación interanual del
3.50\%. A lo largo del año, este índice se mantuvo por encima del límite
superior de la meta inflacionaria establecida por el Banco Central de
Costa Rica (3\%, ±1 p.p.), con la excepción del mes de enero. El valor
más alto se registró en agosto, alcanzando un 12.13\%, seguido de una
desaceleración en los cuatro meses siguientes, donde en diciembre se
registró una inflación interanual de 7.88\%. La inflación importada fue
la principal causa del aumento de esta variable, lo que provocó un alza
en los precios de los alimentos y los derivados del petróleo. La
situación geopolítica, particularmente la Guerra entre Rusia y Ucrania,
contribuyó a estas presiones inflacionarias, aunque se observó un freno
en ellas a partir del mes de agosto, coincidiendo con una disminución en
el tipo de cambio.

En el primer trimestre del 2023, la inflación general mantuvo la
tendencia a la baja iniciada en septiembre del 2022. Luego de alcanzar
una tasa de variación máxima en agosto del 2022, retrocedió de manera
sostenida hasta ubicarse en marzo del 2023 en 4,4\%. Según el Informe de
Política Monetaria de abril del 2023, el comportamiento de la inflación
en el primer trimestre del presente año es explicado en mayor medida por
los precios de los bienes, cuyos aumentos interanuales se moderaron con
respecto a los trimestres previos.

Durante el segundo trimestre del 2023 se acentuó la reducción de la
inflación general, lo que llevó a que se ubicara por debajo del límite
inferior del rango de tolerancia al final de este periodo. A finales de
junio, la inflación general, se ubicó en -1,0\%. Para el cuarto
trimestre del 2023, al igual que en el previo, la inflación general
interanual fue negativa (deflación. En diciembre del 2023, la inflación
general se ubicó en -1,8\%. Según el Informe de Política Monetaria de
Enero del 2024, la reducción de la inflación, con valores negativos
desde junio del 2023, ha estado determinada, en mayor medida, por la
reversión de los choques de oferta externos y la política monetaria
restrictiva.

En cuanto a las expectativas para el año 2024, se espera un repunte
gradual en la inflación en Costa Rica, aunque se mantendría dentro de
los objetivos establecidos por el Banco Central. Este repunte se
atribuye en gran medida al comportamiento de las materias primas
importadas. Se proyecta que la inflación alcance alrededor del 1,9\%
(interanual) para ese año, marcando una salida del periodo de deflación
en el que se encuentra actualmente. Sin embargo, el momento exacto en
que se dará este cambio aún no está determinado. A pesar de este
repunte, el país podría cerrar el año nuevamente fuera del rango
objetivo del Banco Central, según proyecciones de la OCDE. Se espera
que, para el año siguiente, 2025, la inflación regrese al rango
objetivo, ubicándose en torno al 3,1\%, reflejando una mejora en las
condiciones económicas que podría generar presiones inflacionarias
internas.

\hypertarget{tasas-de-interuxe9s-experimentadas-y-esperadas}{%
\section{Tasas de interés experimentadas y
esperadas}\label{tasas-de-interuxe9s-experimentadas-y-esperadas}}

\hypertarget{tasas-de-interuxe9s-experimentadas}{%
\subsection{Tasas de interés
experimentadas}\label{tasas-de-interuxe9s-experimentadas}}

\hypertarget{tasas-de-interuxe9s-esperadas}{%
\subsection{Tasas de interés
esperadas}\label{tasas-de-interuxe9s-esperadas}}

\hypertarget{producto-interno-bruto}{%
\section{Producto Interno Bruto}\label{producto-interno-bruto}}

\includegraphics{Informe-Contexto-Económico_files/figure-latex/unnamed-chunk-6-1.pdf}

\includegraphics{Informe-Contexto-Económico_files/figure-latex/unnamed-chunk-7-1.pdf}

Tras la pandemia del COVID-19, en el 2022 la economía del mundo se
comenzó a recuperar debido a las campañas de vacunación contra el virus
y en consecuencia la eliminación de muchas de las restricciones
sanitarias que impedían el desenvolvimiento económico. Esto se puede
observar por medio del gráfico que muestra la evolución del PIB en donde
para cada uno de los trimestres del año 2022 existe un crecimiento.

Al estudiar la variación del PIB entre el primer trimestre del 2021 y el
primero del 2022 se obtiene la mayor variación interanual del PIB para
el periodo de estudio con relación a la reapertura de la economía del
país. Se puede notar que en los próximos trimestres del año las
variaciones interanuales son positivas, no obstante, mucho menores que
al inicio.

En el primer trimestre del 2023 se tuvo un decremento en el PIB, sin
embargo, al analizar la variación interanual se afirma que este tuvo un
crecimiento del 3,99\%. El resto del año, al hablar de cantidades, se
mantuvo en aumento. Con lo que respecta al cambio del trimestre
respectivo del año anterior este decrece durante todo el año. Es
importante recalcar que según el informe de política monetaria publicado
por el BCCR se proyecta una desacelerecaión en el Producto Interno Bruto
hasta alcanzar el 4\% anual en el 2024 y 2025.

\hypertarget{variaciuxf3n-de-salarios}{%
\section{Variación de Salarios}\label{variaciuxf3n-de-salarios}}

\includegraphics{Informe-Contexto-Económico_files/figure-latex/unnamed-chunk-13-1.pdf}

Según el informe de la Organzación Mundial del Trabajo (OIT), entre
enero y octubre del 2022 el salario mínimo real de Costa Rica en
promedio se desvalorizó en un 4,7\% ya que, como se puede observar en el
gráfico anterior en donde se refleja que si bien las variaciones
nominales muestran una estabilidad positiva y un aumento en julio del
2022, las variaciones reales son todo lo contrario pues, cada mes se da
un aumento de manera negativa hasta en agosto de ese año donde empiezan
a disminuir. Hasta enero del 2023 la variación real logra un valor
positivo de 0,38\%, a partir de ese momento mantuvo una tendencia a la
alza hasta agosto del mismo año.

Por otro lado, las variaciones nominales en el 2023 continuaron
creciendo hasta julio en donde decrecen a 6,74\%, un mes antes por
primera vez durante el periodo de estudio las variaciones reales
sobrepasan las nominales.

El informe indica que la tendencia a la baja del salario real en el 2022
se debe al aumento de la inflación para este periodo y como se analiza
en el apartado correspondiente a esta variable por medio del gráfico se
evidencia la clara relación inversa existente entre estas dos variables,
justo en agosto de 2022 la inflación toma su valor más alto superando al
12\%, misma fecha en la que el salario real toma su valor más bajo de
-10,35\%. Mientras que, para el 2023 la inflación sufre de una reducción
importante llegando a una deflación, por consecuencia el salario real se
recupera.

\hypertarget{uxedndices-de-precios-al-consumidor}{%
\section{Índices de Precios al
Consumidor}\label{uxedndices-de-precios-al-consumidor}}

\hypertarget{duxe9ficit-fiscal}{%
\section{Déficit fiscal}\label{duxe9ficit-fiscal}}

\hypertarget{tipo-de-cambio}{%
\section{Tipo de cambio}\label{tipo-de-cambio}}

\begin{Shaded}
\begin{Highlighting}[]
\NormalTok{TipoCambio1 }\OtherTok{\textless{}{-}} \FunctionTok{read.csv2}\NormalTok{(}\StringTok{"TipoCambio1.csv"}\NormalTok{)}
\NormalTok{TipoCambio1 }\OtherTok{\textless{}{-}}\NormalTok{ TipoCambio1[, }\SpecialCharTok{{-}}\FunctionTok{c}\NormalTok{(}\DecValTok{4}\SpecialCharTok{:}\DecValTok{10}\NormalTok{)]}
\NormalTok{TipoCambio1}\SpecialCharTok{$}\NormalTok{Fecha }\OtherTok{\textless{}{-}} \FunctionTok{as.Date}\NormalTok{(TipoCambio1}\SpecialCharTok{$}\NormalTok{Fecha, }\AttributeTok{format =} \StringTok{"\%d/\%m/\%Y"}\NormalTok{)}
\end{Highlighting}
\end{Shaded}

\begin{Shaded}
\begin{Highlighting}[]
\FunctionTok{ggplot}\NormalTok{(TipoCambio1, }\FunctionTok{aes}\NormalTok{(}\AttributeTok{x =}\NormalTok{ Fecha)) }\SpecialCharTok{+}
  \FunctionTok{geom\_line}\NormalTok{(}\FunctionTok{aes}\NormalTok{(}\AttributeTok{y =}\NormalTok{ TIPO.CAMBIO.COMPRA, }\AttributeTok{color =} \StringTok{"Compra"}\NormalTok{)) }\SpecialCharTok{+}
  \FunctionTok{geom\_line}\NormalTok{(}\FunctionTok{aes}\NormalTok{(}\AttributeTok{y =}\NormalTok{ TIPO.DE.CAMBIO.VENTA, }\AttributeTok{color =} \StringTok{"Venta"}\NormalTok{)) }\SpecialCharTok{+}
  \FunctionTok{scale\_color\_manual}\NormalTok{(}\AttributeTok{values =} \FunctionTok{c}\NormalTok{(}\StringTok{"Compra"} \OtherTok{=} \StringTok{"\#2F4F4F"}\NormalTok{, }\StringTok{"Venta"} \OtherTok{=} \StringTok{"cadetblue3"}\NormalTok{)) }\SpecialCharTok{+}
  \FunctionTok{labs}\NormalTok{(}\AttributeTok{x =} \StringTok{"Fecha"}\NormalTok{, }
       \AttributeTok{y =} \StringTok{"Tipo de Cambio (colones)"}\NormalTok{,}
       \AttributeTok{title =} \StringTok{"Tipo de cambio del dólar de los Estados Unidos de América"}\NormalTok{, }
       \AttributeTok{subtitle =} \StringTok{"Ene 2022 a Dic 2023"}\NormalTok{,}
       \AttributeTok{color =} \StringTok{"Tipo de Cambio"}\NormalTok{,}
       \AttributeTok{caption =} \FunctionTok{expression}\NormalTok{(}\FunctionTok{bold}\NormalTok{(}\StringTok{"Fuente: "}\NormalTok{) }\SpecialCharTok{*} \StringTok{"Elaboración propia a partir de datos del BCCR."}\NormalTok{))}\SpecialCharTok{+}
  \FunctionTok{scale\_y\_continuous}\NormalTok{(}\AttributeTok{limits =} \FunctionTok{c}\NormalTok{(}\DecValTok{500}\NormalTok{, }\DecValTok{700}\NormalTok{),  }\CommentTok{\# Establece los límites del eje y}
                     \AttributeTok{breaks =} \FunctionTok{seq}\NormalTok{(}\DecValTok{500}\NormalTok{, }\DecValTok{700}\NormalTok{, }\AttributeTok{by =} \DecValTok{20}\NormalTok{))}\SpecialCharTok{+}
   \FunctionTok{scale\_x\_date}\NormalTok{(}\AttributeTok{date\_labels =} \StringTok{"\%b{-}\%Y"}\NormalTok{, }\AttributeTok{date\_breaks =} \StringTok{"2 month"}\NormalTok{)}\SpecialCharTok{+}
  \FunctionTok{theme}\NormalTok{(}\AttributeTok{axis.text.x =} \FunctionTok{element\_text}\NormalTok{(}\AttributeTok{angle =} \DecValTok{45}\NormalTok{, }\AttributeTok{hjust =} \DecValTok{1}\NormalTok{),}
        \AttributeTok{legend.position =} \StringTok{"bottom"}\NormalTok{,}
        \AttributeTok{plot.caption.position =} \StringTok{"plot"}\NormalTok{,}
        \AttributeTok{plot.caption =} \FunctionTok{element\_text}\NormalTok{(}\AttributeTok{hjust =} \DecValTok{0}\NormalTok{),}
        \AttributeTok{plot.title =} \FunctionTok{element\_text}\NormalTok{(}\AttributeTok{hjust =} \FloatTok{0.5}\NormalTok{),}
        \AttributeTok{plot.subtitle =} \FunctionTok{element\_text}\NormalTok{(}\AttributeTok{hjust =} \FloatTok{0.5}\NormalTok{))}
\end{Highlighting}
\end{Shaded}

\begin{verbatim}
## Warning: Removed 62 rows containing missing values or values outside the scale range
## (`geom_line()`).
## Removed 62 rows containing missing values or values outside the scale range
## (`geom_line()`).
\end{verbatim}

\includegraphics{Informe-Contexto-Económico_files/figure-latex/unnamed-chunk-15-1.pdf}

En diciembre de 2022, el tipo de cambio cerró con una tasa de compra de
\(₡594/US\$\) y una tasa de venta de \(₡601/US\$\). Estas cifras fueron
notablemente más bajas en comparación con los niveles observados a lo
largo del año, el cual estuvo caracterizado por una alta volatilidad en
esta variable. Durante el año, el tipo de cambio experimentó una
apreciación del 7\%. Inicialmente, el tipo de cambio mostró una
tendencia al alza en los primeros meses, alcanzando su punto más alto a
mediados del año, sin embargo, experimentó un alto en el segundo
semestre tras el anuncio por parte del Banco Central de Costa Rica
(BCCR) de una serie de medidas destinadas a controlar esta tendencia.
Como resultado, en el segundo semestre del año, el tipo de cambio
comenzó a descender. Entre las medidas adoptadas por el BCCR para
contrarrestar el alza en el tipo de cambio se encuentran la emisión de
instrumentos financieros en dólares, la reducción del horario de
negociación del MONEX y el otorgamiento de un préstamo de \$1 000
millones al Fondo Latinoamérica de Reservas (FLAR) con el fin de
fortalecer las reservas del país.

Durante el 2023, el tipo de cambio en Costa Rica se promedió en
₡544,5/US\$, alcanzando un nivel de precios que no se había visto desde
el año 2016. Este valor incluso superó la cifra registrada hace 14 años,
en 2009. Según análisis realizados, la devaluación de la moneda alcanzó
el 12\% a lo largo del año, con un promedio de compra del dólar de 542
colones y venta de 548 colones. Este exceso de oferta de dólares se
atribuye principalmente al aumento en la llegada de turistas al país,
que alcanzó los 2,75 millones de visitantes extranjeros hasta septiembre
de 2023, según datos del Instituto Costarricense de Turismo, y al
crecimiento en la Inversión Extranjera Directa (IED), que registró un
incremento del 16\% hasta septiembre del 2023, representando la llegada
al país de 2.691 millones de dólares, según el Ministerio de Comercio
Exterior. Ante la posibilidad de una baja más pronunciada en el valor
del dólar, el Banco Central intervino en el mercado mediante la compra
de divisas en Monex, con el fin de satisfacer la demanda del sector
público no financiero y aumentar las reservas monetarias
internacionales.

Para el año 2024, los analistas financieros proyectan una intervención
cambiaria más cautelosa por parte del Banco Central, limitada a
situaciones de sobreoferta de dólares. La expectativa de una inflación
del 2\% sugiere una estabilidad en el tipo de cambio. No se prevén
eventos en Costa Rica que puedan ejercer presiones significativas al
alza sobre el tipo de cambio, lo que lleva a proyecciones de un tipo de
cambio estable. Además, la próxima recepción de \$1 mil millones en
Eurobonos también se considera en estas previsiones.

\end{document}
